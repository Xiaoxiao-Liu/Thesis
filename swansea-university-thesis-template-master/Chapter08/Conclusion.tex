%-----------------------------------------------------------------------------------------
\section{Conclusion}
%-----------------------------------------------------------------------------------------
In conclusion, data visualisation is becoming a fundamental necessity in various domains due to the need for understanding increasingly large volumes of information. This project, which aims at developing an interactive visualisation system, reveals that there is great potential in the ways that data visualisation can be exploited in information analysis. 

The first aim for this project was to create a parallel text visualisation to assist users comparing words in different translation versions. This aim is met with the four visualisations:
\begin{itemize}
	\item \textbf{} Visualise the most frequent words in each version. This has been achieved in the Frequency Visualisatoin.
	\item \textbf{} Visualise the most important words in each verson. This feature can be seen from the Tf-Idf Visualisation.
	\item \textbf{} Visualise the most frequent dictionary words in each version, which is provided in Lemma and Frequency Visualisation.
	\item \textbf{} Visualise the most important dictionary words in each version. This visualisation is displayed in the Lemma and Tf-Idf Visualisation.
\end{itemize}
All these four visualisations have been accomplished in this project. However, the performance of the visualisations remain to be improved in the future. For example, all the four visualisations are parallel texts, so the Frequency Visualisation can be combined into one of other three visualisations.  

The second aim is to provide a software application which enables interactive user options. This is also met by rendering following features:
\begin{itemize}
	\item \textbf{Button} used to toggle the visualisations. 
	\item \textbf{Slider} used to scale the visualisation.
	\item \textbf{Slider} used to scale the frame.
	\item \textbf{Button} used to turn on and off the texts in the visualisation.
	\item \textbf{Menu} used to select certain concordances. It also can be used to arrange the order of concordances.
	\item \textbf{Interactive Colour Legend} used to visualise frequency or Tf-Idf values of the words. It also can be used to view words with certain values.
	\item \textbf{Interactive blocks} used to view specific words and the same words in other versions by highlighting these words.	
\end{itemize}

Overall, the project can be deemed to be a success. However, there are some issues with the performance of the algorithm when version selection is attempted in Lemma visualisation. A limitation with highlighting items has also been identified if the slider bars are notmoved.

A video demonstration of the software is available along with other resources produced for the project at the following web address: \url{http://cs.swansea.ac.uk/~cswang/Xiaoxiao_Othello/Doxygen/}
 


