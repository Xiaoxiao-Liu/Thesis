%-----------------------------------------------------------------------------------------


\section{Introduction and Motivation}
%-----------------------------------------------------------------------------------------

The number of data sets have risen dramatically over the past few years, and these data sets have become increasingly complicated to analyse.  Dealing with large volumes of data has become a challenge in certain fields\cite{Larameea}. According to \cite{Ward2015}, when receiving large volumes of information, people tend to use sight as the main sense to understand it. Data visualization, is a mechanism for using graphics to represent data \cite{Ward2015}, which provides a good solution for exploring complex data.

As stated by \cite{Williams1995}, data visualization is defined as “the visual representation of a domain space using graphics, images, animated sequences, and sound augmentation to present the data, structure, and dynamic behavior of large, complex data sets that represent systems, events, processes, objects and concepts”\cite{Williams1995}. Data visualization allows us to explore large complex data sets efficiently, gaining insights we might otherwise have missed.

Text data is generated in large quantities every day by newspapers, blogs, and social media. Extracting information from text data is becoming increasingly important. In some areas, studying the relationship between words, sentences and texts’ structure may help researchers to understand important information hiding in the text. For example, analysing the text found from a historic site may help understand the dates of files, antecedent events, or the host of the grave, even without an understanding of the ancient language. Similarly, in the archaeological industry, techniques in text data analysis are fundamental and significant in translation study. Many institutes rely on knowledge of text data analysis to explore the variation of language in history, style of authors, as well as the social status of people in a particular time period.

Methods to analyse and present text data have become a popular topic as the volume of the text data is often huge and complicated in format, genre, and morphology. For instance, languages inherited from different roots may lead to different expressions when translating from one to another. Authors from different eras or regions may use different words to express the same things. The same contents may appear in different styles of expressions according to the purpose of the texts.
Also, to deal with these problems, text data can be analysed and represented from lexical, syntactic and semantic perspectives \cite{Ward2015}, so that the unstructured text can be converted to structured data. Calculating frequency and weights of words can help to explore the information of content. There have been plentiful tools to visualize the structure of text data, such as Word Clouds, Word Tree, Tex Arc,  etc. Text data is often analysed separately in a single document and a collection of documents. One such collection of documents is \emph{Othello}.

\emph{Othello}, as one of the greatest tragedies of Shakespeare's plays, has been translated more than 60 times in German  \cite{Geng2011}.The College of Arts and Humanities at Swansea University has a collection of 55 different German translations of \emph{Othello}. The time span of these translations ranges from 1766 to 2010. There are also different genres such as poems and prose, as well as plays. Applying data visualisation techniques to help represent this text data can contribute to new insights of Shakespeare’s work, and explorations into visualization. More concretely, the aims of this project are as follows:

\begin{itemize}
	\item \textbf{}To develop an interactive visualization system which enables the researchers in the College of Art and Humanities to explore detailed translation information of different translation versions.
	\item \textbf{} To design textual data visualization algorithms to display more information by comparing different versions of translations, such as time span, genre, interpretation.
	\item \textbf{} To explore potential solutions in textual data visualization for difficulties in translation comparison, such as parallel text and data filtering.
\end{itemize}

Using textual data visualization as an aid to explore the text data of \emph{Othello}’s translations will  benefit for researchers in understanding the changes, interactions, and impacts of cultures, time span, and styles \cite{Alrehiely2014}. Based on the work of \cite{Geng2015}, \cite{Alrehiely2014}, and \cite{Tom2012}, we attempt to develop an interactive visualization system aimed to allow our users to view, compare, and analyse tokens in each version. The visualisation tool will be designed to assist in viewing the variation of tokens in different translation versions, and in comparing the varieties of tokens after applying different methods to process the text data.  Apart from the essential information about each version, such as the author and date of publication, there are three unique fields the data provides: the frequency of tokens, weight of tokens, and results from lemmatization for tokens.

The outcomes of the visualization system should be helpful in understanding the variation of word morphologies, varieties of text styles, and the complex features of the German language. It also facilitates improved comprehension of literature dynamics, the differences between languages, and the perception of translating cultures. Moreover, this project will provide a visualization tool for books, articles, newspapers, etc., to represent large sets of text data.  

Several special problems are caused by the German Shakespeare text data used in this project, such as antiquated language, and poetic orthography.  The former means that some words used in the 18th or 19th century may not be in the lexis of training corpora if these are based on 20th/21st century sources. Poetic orthography can be explained using the example of “verloren" (meaning: lost), the word is normally written as “verloren", though can also be spelled verlor’n, or verlorn in some places in Shakespeare texts (the word normally has 3 syllables, pronounced VER-LOR-RUN, but the writer wants it to be spoken as 2 syllables, VER-LORN). This happens a lot, yet there are no effective algorithms to recognise these forms. To find a solution to these problems some methods from Natural Language Processing may be applied, such as lemmatization.


Choosing this project was on account of my interest in the field of data visualization and language analysis. The background of programming and language study will further my comprehension of data analysis and processing. Developing a project such as Translation Visualisation is becoming a significant topic for language studying and text data processing.

Following \cite{Laramee2011}, the rest of this thesis is structured as follows: Section 2 details the background research, along with the literature review, introduction to existing systems, and data characteristics. Section 3 details the specifications of the project, which includes the features specification of software and technology choices. Section 4 presents the approach of the project, time arrangement and potential risks. Section 5 provides an overview of project design. Section 6 describes how the project is implemented. In section 7, we provide the performance and feedback from a domain expert as an evaluation. Section 8 draws a conclusion of this project, and section 9 discusses potential further work. 

