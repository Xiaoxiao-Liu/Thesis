%-----------------------------------------------------------------------------------------

\section{Introduction and Motivation}
%-----------------------------------------------------------------------------------------

Data sets hava risen dramatically over the past few years, and these data sets have become increasingly complicated to analyse. How to deal with large amounts of data has become a challenge in certain fields\cite{Larameea}. According to \cite{Ward2015}, when receiving large volumes of information, people tend to use sight as the main sense to understand it. Data visualization, as a mechanism  using graphics to represent data \cite{Ward2015}, provides a good solution in exploring huge sets of complicated data.
As stated by \cite{Williams1995}, data visualization is defined as "the visual representation of a domain space using graphics, images, animated sequences, and sound augmentation to present the data, structure, and dynamic behaviour of large, complex data sets that represent systems, events, processes, objects and concepts"\cite{Williams1995}. By applying techniques of data visualization, more information can be explored.

Text data emerges in large quantities every day in newspapers, blogs, and social media. Hence, exacting information from text data is becoming highly needed. In certain study area, studying the relationship between words, sentences and texts’ structure may help researchers to understand important information hiding in the text. For example, in archaeologists’ lab, analysing the text they found from historic site may help them understand the dates of files, events happened, or the host of the grave, even without knowing the meaning of the ancient language. Similarly in the archaeological industry, techniques in text data analysing is fundamental and significant in translation study. Many institutes rely on knowledge of text data analysis to explore the variation of language in history, style of authors, as well as the life status people in particular period of time. 

The ways to analyse and present text data has become a popular topic as the volume of the text data is often huge and complicated in format, genre, and morphology. For instance, languages inherited from different roots may lead to different expressions when translating from one to another. Authors of different eras or regions may use different words to express the same things. Same contents may appear in different style of expressions according to the purpose of texts. Also, to deal with these problems, text data can be analysed and represented from lexical, syntactic and semantic perspective \cite{Ward2015}, so that the unstructured text can be converted to structured data. Calculating frequency and weights of words can help to explore the information of content. There have been plentiful tools to visualization the structure of text data, such as Word Clouds, Word Tree, Tex Arc, etc. And for different research purpose, text data are often analysed separately in single document and a collection of documents. One such collection of documents is \emph{Othello}.

\emph{Othello}, as one of the greatest tragedies of Shakespeare's play, are translated more than 60 times in German by now \cite{Geng2011}.The College of Arts and Humanities at Swansea University has a collection of 57 different German translations of \emph{Othello}. The time span of these translation is from 1766 to 2010. And there are also different genres such as poems, prose, as well as plays. Applying data visualization techniques to help to represent these text data will contribute to new research in Shakespeare’s work studying, and visualization exploring. More concretely, the aims of this project are as follows:

\begin{itemize}
	\item \textbf{1} To develop an interactive visualization system that enable the researchers in the College of Art and Humanities to explore detailed translation information of different versions. .
	\item \textbf{2} To design a software of textual data visualization to display more information by compare different versions of translations, such as time span, genre, interpretation.
	\item \textbf{3} To explore potential solutions in textual data visualization for difficulties in translation comparison, such as parallel text and data filtering.
\end{itemize}

Using textual data visualization as an assistance to explore the text data of \emph{Othello}’s translations will benefit for researchers to understand the changes, interactions, and impacts of these translation versions and cultures, time span, styles\cite{Alrehiely2014}. Based on the work of \cite{Geng2015}, \cite{Alrehiely2014}, and \cite{Tom2012a}, we attempt to develop an interactive visualization system aims to allow our users to view, compare, and analyse tokens in each version. The visualization tool will be designed to assist in viewing variation of tokens in different translation versions, and in comparing the varieties of tokens after applying different methods to process the text data. Apart from the essential information about each version, such as the author, data of publication, there are three unique information of data are provided: the frequency of tokens, weight of tokens, and results from lemmatization for tokens. 

The outcomes of the visualization system should be helpful to understand of the variation of word morphologies, varieties of text styles, as well as the complex features of German language. It also contributes to comprehend the dynamics of literature, the differences between language, and the perception of translating cultures. Moreover, this project will provide a visualization tool for books, articles, newspapers, etc., to represent large sets of text data.  

However, there exists some challenges in this project. As a highly inflected language, German is featured as complex grammar structure and numerous compound words. With the German Shakespeare text data in this project, several special problems are caused by antiquated language, and poetic orthography. The former means some words used in the 18th or 19th century may not be in the lexis of training corpora, if these are based on 20th/21st century sources. And by using the poetic orthography , take “verloren” (meaning: lost) for example, the word normally written as “verloren”, also can be spelled verlor'n, or verlorn in some places in Shakespeare texts (the word normally has 3 syllables, pronounced VER-LOR-RUN, but the writer wants it to be spoken as 2 syllables, VER-LORN). This kind of situation happens a lot. Yet there’s no effective algorithms to recognise these forms. To find a solution for for these problems, some methods from Natural Language Processing may be applied, such as lemmatization.

Choosing this project for my dissertation was account of my interest in the field of data visualization and language analysis. The background of programming and language studying further my comprehension in data analysis and processing. Developing a project such as Translation Visualisation is becoming a significant topic for language studying and text data processing.

Following \cite{Laramee2011}, the rest of this paper is structured as follows: Section 1 to section 4 are modified versions of work previously presented by the author in \cite{Liu}. Section 2 details the background research, including literature review, introduction to existing systems, and data characteristics. Section 3 is the specification of the project which includes the features specification and technology choices to the software. Section 4 presents the approach of the project, time arrangement and potential risks. Section 5 provides an overview of project design.  Section 6 describes how the project is implemented, including the basic features of the software, and the enhancement features.In section 7, we provide the performance and feedback from a domain expert as evaluation. And section 8 draws a conclusion of this project and section 9 discusses the potential further work. 



