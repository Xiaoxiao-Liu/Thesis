%-----------------------------------------------------------------------------------------
\clearpage
\section{Background Research}
%-----------------------------------------------------------------------------------------

In this section, a literature review is firstly introduced to present the most relevant works to this project. In the second part, previous systems in similar project is introduced. The third part provides an detailed analysis of the data characteristics.

\subsection{Literature Review}
This section introduced principles and techniques for data preprocessing and text data visualization.

Text Visualisation Browser \cite{Kucher2014} is an online tool providing the most comprehensive summary of published text visualization \cite{Cao2016}. According to Text Visualisation Browser, from 1976 to 2017, there are 400 published text visualisation papers in total, in which 396 publications are aim to analyse text alignment. By searching "Word", there shows 20 publications, and "Translation" gets 16 results. Whereas when typing "Frequency" and "Weighting", each key word get 1 results. Also, key words such as "Machine learning, "Data Mining", "Natural Language Processing" got no publication collected. The results indicate that in text visualization domain, most researches focus on presenting alignment of texts. There are certain amounts of research focus on the topic such as "word analysis" and "translation", which is similar with this project. However, applying more specific techniques such as "Natural Language Processing" haven't been applied in text visualisation widely.


\paragraph{Interactive Exploration of Versions across Multiple Documents}

\paragraph[]{}In the work of \emph{Interactive Exploration of Versions across Multiple Documents}, \cite{Jong2008} provide an interactive visualization tool, MultiVersioner, to address the issues of comparing several versions of texts. The MultiVersioner enables users to search for items such as words, phrases and lines, along with the analysis of the frequency patterns of these items. In addition, methods such as colour-coded highlighting and overview are also rendered in this tool. Figure \ref{fig:multiVersioner} serves as an example of overview for many versions and documents in this software. In the overview, terms are denoted by blocks. If user mouses over a single block, a tooltip with the relevant sentence will be popped up.

\begin{figure}[h]
	\centering	
	\includegraphics[width=13cm, height=8cm]{Figs/MultiVersioner}\\[1ex]
	\caption{Overview of many versions and documents in MultiVersioner (\cite{Jong2008})}
	\label{fig:multiVersioner}
\end{figure} 

The work of MultiVersioner allows users to compare multiple documents. Meanwhile, it provides a helpful feature to search for entities such as words and lines. Moreover, it can be served as a tool to analyse the frequency patterns of the words. However, there are only limited features provided in this software. Some helpful functions such as alignments between versions, or version turning on and off need to be explored.

\paragraph{Interactive Visual Alignment of Medieval Text Versions}
\paragraph[]{}

\cite{Stefan2017} discussed a novel methods to compare text versions in the work \emph{Interactive Visual Alignment of Medieval Text Versions}. They provide a visual analytics system which enables computationally align complex textual differences such as orally inflected text. The data they deal with is a group of medieval poetries with complex text forms. Their works includes three basic visualisations:

\begin{itemize}
	\item \textbf{A visualisation of text alignment} This is a visualisation in which highly unstable text versions are identified and aligned. This feature is accomplished applying parameter-driven approach. In addition, a visual analytics procss is rendered which accept tweaking parameters for iterative improvement of the alignment.
	\item \textbf{Multi-level alignment visualisation} In this view, various visualisations are presented which enables users to analyse texts alignments on different hierarchy levels.
	\item \textbf{Meso Reading} This is a visualisation to interpret texts in parallel. Meanwhile the connections are shown among text versions. This feature is considered as an novel feature which provides an intermediate perspective to display complex variance in texts.
	
Figure \ref{fig:mesoReading} is an interpretation of this project. The screen shot in \cite{Stefan2017} displays different views of the distant reading, the meso reading, the close reading, and full-line matches. In part one, distant reading results of high similarity on words are displayed in the form of parallel lines. In addition, diagonal lines represents several repetitions. In part two, meso reading results are shown in the verse line 'Qui me dist que li ange sont'. In part three, a close reading feature is explored to show false positive alignment while part four are the matches for numerous full-line text.
	
	\begin{figure}[h]
		\centering	
		\includegraphics[width=13cm, height=8cm]{Figs/Meso-Reading}\\[1ex]
		\caption{a demonstration of a result from visual analytics system (\cite{Stefan2017})}
		\label{fig:mesoReading}
	\end{figure} 

	
\end{itemize}

However, this tool is more suitable for French context.

\paragraph{Visualizations Translation Variation of Shakespeare's \emph{Othello}: A Survey of Text Visualisation and Analysis Tools}
\paragraph[]{} In their work \emph{Visualizations Translation Variation of Shakespeare's Othello: A Survey of Text Visualisation and Analysis Tools}, \cite{Geng2011} developed a visualisation system which can be used to view and analyse variations between translation and based text. The data of this project is a collection of German translations of Shakespeare's \emph{Othello}. In this project, several techniques are applied to get an interactive visualisation system. These techiniques include parallel coordinate, Tree map, and DOI-tree. The tools they developed provides features that enables users to brush words so that these words can be displayed in  a parallel tag cloud. Figure \ref{fig:tagCloud} is a screen shot which shows the outcome of applying the TagCrowd feature into one of Othello translation version. In this view, the stop words are identified and removed manually from the origianl text.

\begin{figure}[h]
	\centering	
	\includegraphics[width=13cm, height=8cm]{Figs/Tagcloud}\\[1ex]
	\caption{A TagCrow (\cite{TagCrowd}) visualisation of a passage from \emph{Othello}. }
	\label{fig:tagCloud}
\end{figure} 

\paragraph{ShakerVis: Visual Analysis of Segment Variation of German Translations of Shakespeare’s \emph{Othello}}
\paragraph[]{} ShakerVis \cite{Geng2015} is a special visualisation tool which is designed to provide an interactive visualisation system to display version variations. In this system, \cite{Geng2015} applies following visualisation techiniques:

\begin{itemize}
	\item \textbf{Parallel Coordinate View} provides outcomes by using Eddy value. The description of Eddy value can be seen in Previous System chapter.
	\item \textbf{Scatter Plot View} 
	\item \textbf{}
	\item \textbf{}
	\item \textbf{}
	\item \textbf{}
	
\end{itemize}

\subsection{Previous Systems}

The Version Variation Visualization (VVV) project was introduced by Dr Tom Cheesman from Modern Language Centre at Swansea University. It aims to create interactive data visualization system to build cross-cultural exploration networks. The VVV project focus on developing digital tools which can help to compare and analyze different versions of translation \cite{Cheesman2012}. So far, the tools developed in the project is Ebla, Prism and ShakerVis. Ebla, served as the copus, is a software to stock the text data and detailed information of them. Prism provides the interface for separating texts into segments and processing the segments as alignment. Based on the idea of these two software, ShakerVis provides an interactive interface for visualizing the information of the translation versions \cite{Geng2015}.

There are three types of data visualization in this project: Time-Map, Alignment Maps, Parallel view and Eddy and Viv view. 

\paragraph{Time-Map}
\paragraph[]{}

Figure \ref{fig:timeMap} provides a screen shot of Time Map, which shows the location of the authors and the year of translation versions published. From this view, we can tell that some particular places such as Berlin and Dresden

\begin{figure}[h] 
	\centering	
	\includegraphics[width=16cm, height=14cm]{Figs/Time-Map}\\[1ex]
	\caption{Time-Map provides an interactive overview of the corpus meta data (\cite{Cheesman2012})}
	\label{fig:timeMap}
\end{figure} 

\paragraph{Alignment Map}
\paragraph[]{}

Figure \ref{fig:alignmentMap} exhibits a structure visualization which compares the segments of the texts between base text and translation versions. By comparing these texts, one can tell the general difference and variation between the base text and translations. For example, if one paragraph of several translations is longer than that of the base text, it is possible that particular expression of German is more complex or detailed than the English. 

\begin{figure}[h] 
	\centering	
	\includegraphics[width=16cm, height=14cm]{Figs/Alignment-Map}\\[1ex]
	\caption{Alignment Maps provides an comparative visualiztion of segments (\cite{Cheesman2012})}
	\label{fig:alignmentMap}
\end{figure} 

\paragraph{Parallel View}
\paragraph[]{}

Figure \ref{fig:parallelView} shows a straightforward view between base text and selected translations. In this visualization, segments are more explicit to find.

\begin{figure}[h] 
	\centering	
	\includegraphics[width=16cm, height=14cm]{Figs/Parallel-View}\\[1ex]
	\caption{Parallel View provides an explicit view between the base text and selected version (\cite{Cheesman2012})}
	\label{fig:parallelView}
\end{figure} 

\paragraph{Eddy and Viv View}
\paragraph[]{}

Figure \ref{fig:eddyVivView} shows Eddy and Viv view, which provides more information of the translation comparison. From the sort bar, we can tell that there are four types can be visualized. Eddy value shows the variation of words used in segment. Relatively, Viv value provides the changes or rivalries for some segments in translation. If we choose version name, segment length or reference date as the order of sorting, there will be other information of translation variations. Also, there are backtranslation based on machine translation provided, which is another powerful function for comparing the text data.

\begin{figure}[h] 
	\centering	
	\includegraphics[width=16cm, height=14cm]{Figs/Eddy-Viv-View}\\[1ex]
	\caption{Eddy and Vis view enable researchers to understand more details of vocabulary (\cite{Cheesman2012})}
	\label{fig:eddyVivView}
\end{figure} 



