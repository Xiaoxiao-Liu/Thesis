%-----------------------------------------------------------------------------------------
\section{Future Work}
%-----------------------------------------------------------------------------------------
The first priority to continue this project would be to lemmatise the English version text. The project for now is only supported lemma views for German lemmatisation, with English version still keep many inflected words such as 'I', 'my', me. Further more, if weightings for these English words calculated in specific contexts is rendered, this project may display new results. For the English lemmatisation, this can be accomplished by introduced extra English lemmatisation libraries, such as Stanford NLP Java library. For the English Tf-Idf, this can be attempted by introduce the results from VVV project, which has a corpus stored all the Tf-Idf values computed in the corpus of all Shakespeare works in English.

The software could also be extended to combine the views into one or two visualisation. For example, to provide two views such as Tf-Idf Visualisation and Lemma and Tf-Idf Visualisation in one visualisation, which would allow users to compare different results. In order to do this, new frame can be added to enable the comparison view.

Further improvements could be made to the software such as enabling users to search words by adding a 'search box'. This can be worked out by applying 'Text Field' in Java.
